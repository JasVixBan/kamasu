\documentclass{beamer}
\usepackage{pgfpages}
\usepackage{wrapfig}

\setbeameroption{show notes}
\setbeameroption{show notes on second screen=bottom}

\mode<presentation>
{
  \usetheme{resophonic}
  \setbeamercovered{transparent}
%  \setbeamertemplate{alerted text begin}{hi hi hi}
%  \setbeamertemplate{alerted text end}{end end end}
}

\def\footertext#1{\hbox{\vbox to \logobarheight{
      \vfill\hbox{{\usebeamerfont{logobar}\usebeamercolor[fg]{logobar}#1}}\vfill}}}

% \def\rnote<#1>#2{\note<#1>{\whitesheets #2}}
% \newcommand{\rnote}[2]{ \note#1{#2}}

\addlogo{\footertext{Boostcon 2009}}
\logobartext{Troy D.~Straszheim}

\usepackage{multimedia}

\usepackage[english]{babel}

\usepackage[latin1]{inputenc}

\usepackage{times}
\usepackage[T1]{fontenc}

% \addlogo{blah}
% \def\logobarlogo{hi}

\title[resophonic::kamasu]{\texttt{resophonic::kamasu}}

\subtitle{Computing on the GPU with CUDA and boost::proto}

\author{Troy D. Straszheim}

\institute[Resophonic Systems, Inc.] % (optional, but mostly needed)
{
  Resophonic Systems, Inc.\\
  Washington, DC
}
% - Use the \inst command only if there are several affiliations.
% - Keep it simple, no one is interested in your street address.

\date[BoostCon 2009] % (optional, should be abbreviation of conference name)
{BoostCon, 2009\\
Aspen, CO}
% - Either use conference name or its abbreviation.
% - Not really informative to the audience, more for people (including
%   yourself) who are reading the slides online

\subject{Computer Science}
% This is only inserted into the PDF information catalog. Can be left
% out. 



% If you have a file called "university-logo-filename.xxx", where xxx
% is a graphic format that can be processed by latex or pdflatex,
% resp., then you can add a logo as follows:

% \pgfdeclareimage[height=0.5cm]{university-logo}{university-logo-filename}
% \logo{\pgfuseimage{university-logo}}



% Delete this, if you do not want the table of contents to pop up at
% the beginning of each subsection:
\AtBeginSubsection[]
{
  \begin{frame}<beamer>{Outline}
    \tableofcontents[currentsection,currentsubsection]
  \end{frame}
}


% If you wish to uncover everything in a step-wise fashion, uncomment
% the following command: 

%\beamerdefaultoverlayspecification{<+->}

\begin{document}


\begin{frame}
   \note{honored to be here,

    First read modern c++ design in about 2002 or 3, wrote everything
    with policy classes for awhile, still get a thrill out of it

    love this kind of thing as it gives me an opportunity to look in
    to the minds of people that are smarter than I am, and not in
    real-time, and in ascii, so there's little danger that I'll need
    eyebleach

    last year at boostcon, hartmut was encouraging

    i thought great, opportunity to learn proto, look into eric
    niebler's mind

    CUDA was getting buzz

    opportunity to do both, and for the next hour or so I'm going to
    tell you what happened.  these are the proceeds thus far.

    haven't really discussed what I'm doing with anybody yet, so I
    expect you to see lots of things I'm overlooking, looking forward
    to what happens after this talk, lots of time for questions and
    discussion.

    Not a finished product.
   }
   \titlepage
\end{frame}
%   }
% 
\begin{frame}{Outline}
  \tableofcontents
  % You might wish to add the option [pausesections]
\end{frame}

\section{CUDA}
\subsection{lots of processors}

\section{boost::proto}
\subsection{transforms}

\section{resophonic::kamasu}
\subsection{benchmarks}

\end{document}


